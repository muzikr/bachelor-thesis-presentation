%%%%%%%%%%%%%%%%%%%%%%%%%%%%%%%%%%%%%%%%%
% Focus Beamer Presentation
% LaTeX Template
% Version 1.0 (8/8/18)
%
% This template has been downloaded from:
% http://www.LaTeXTemplates.com
%
% Original author:
% Pasquale Africa (https://github.com/elauksap/focus-beamertheme) with modifications by
% Vel (vel@LaTeXTemplates.com)
%
% Template license:
% GNU GPL v3.0 License
%
% Important note:
% The bibliography/references need to be compiled with bibtex.
%
%%%%%%%%%%%%%%%%%%%%%%%%%%%%%%%%%%%%%%%%%

%----------------------------------------------------------------------------------------
%	PACKAGES AND OTHER DOCUMENT CONFIGURATIONS
%----------------------------------------------------------------------------------------
\documentclass{beamer}
%\documentclass[handout]{beamer}

\usepackage{mathtools}
\usepackage{comment}

\usetheme{focus} % Use the Focus theme supplied with the template
% Add option [numbering=none] to disable the footer progress bar
% Add option [numbering=fullbar] to show the footer progress bar as always full with a slide count

% Uncomment to enable the ice-blue theme
%\definecolor{main}{RGB}{92, 138, 168}
%\definecolor{background}{RGB}{240, 247, 255}

\definecolor{mygreen}{RGB}{0, 128, 0}
\definecolor{myblue}{RGB}{51, 51, 204}
\definecolor{myviolet}{RGB}{255, 0, 255}

\newcommand{\red}[1]{\textcolor{red}{#1}}
\newcommand{\green}[1]{\textcolor{mygreen}{#1}}
\newcommand{\blue}[1]{\textcolor{myblue}{#1}}
\newcommand{\violet}[1]{\textcolor{myviolet}{#1}}

\newcommand{\R}{\mathbb{R}}
\newcommand{\Z}{\mathbb{Z}}
\newcommand{\Rn}[1][n]{\mathbb{R}^{#1}}
\newcommand{\bx}{\textbf{x}}
\newcommand{\by}{\textbf{y}}
\newcommand{\bz}{\textbf{z}}
\newcommand{\bu}{\textbf{u}}
\newcommand{\bv}{\textbf{v}}
\newcommand{\I}{\mathcal{I}}

\newcommand{\floor}[1]{\left\lfloor #1 \right\rfloor}
\newcommand{\floorphi}{\ensuremath{\floor{\phi}\hspace{-0.5ex}}}

\DeclareMathOperator{\E}{\mathbb{E}\,}

%------------------------------------------------

\usepackage{booktabs} % Required for better table rules
%\usepackage{slashbox}
%\usepackage{ulem,centernot}


%----------------------------------------------------------------------------------------
%	 TITLE SLIDE
%----------------------------------------------------------------------------------------

\title{Integer cooperative game theory}

\author{Richard Mužík}

\titlegraphic{\hspace*{5cm}\includegraphics[scale=0.5]{../img/logo-en.pdf}} % Optional title page image, comment this line to remove it

\institute{richard@imuzik.cz}


%------------------------------------------------

\begin{document}

%------------------------------------------------

\begin{frame}
	\maketitle % Automatically created using the information in the commands above
\end{frame}

%----------------------------------------------------------------------------------------
%	 SECTION 1
%----------------------------------------------------------------------------------------

\section{Motivations and introduction} % Section title slide, unnumbered

%------------------------------------------------

\begin{frame}{Game theory - quick reminder}
    \begin{block}{Cooperative game}
        A \textbf{cooperative game} is an ordered pair $(N,v)$, where $N=\{1,\dots,n\}$ is a finite set of players and $v\colon 2^N \to \mathbb{R}$ is the characteristic function. Further, $v(\emptyset) = 0$.
    \end{block}
    \begin{itemize}
        \item $S \subseteq N$ ... coalition
        \item $v(S)$ ... values of coalition
    \end{itemize}
    Goal: \textit{To distribute the value of the game among players}
    \begin{itemize}
	    \item \textbf{Payoff vector} $\bx \in \Rn$
	    \begin{itemize}
	        \item $x_i$ represents payoff of player $i$
	    \end{itemize}
	    \item Vector $\bx \in \Rn$ is \textbf{efficient}, if $\sum_{i \in N}x_i = v(N)$
	    \begin{itemize}
	        \item Usually, we distribute $v(N)$
	    \end{itemize}
	    \item Vector $\bx \in \Rn$ is \textbf{individually rational}, if $x_i \geq v(i)$
	    \begin{itemize}
	        \item players prefer $x_i$ over $v(i)$
	    \end{itemize}
	\end{itemize}
\end{frame}

%------------------------------------------------


%----------------------------------------------------------------------------------------
%	 SECTION 2
%----------------------------------------------------------------------------------------

\section{Definitions} % Section title slide, unnumbered

%------------------------------------------------

\begin{frame}{Discrete cooperative game theory}
    \begin{block}{Integer cooperative game}
        An \textbf{integer cooperative game} is an ordered pair $(N,v)$, where $N=\{1,\dots,n\}$ is a finite set of players and $v\colon 2^N \to \mathbb{Z}$ is the characteristic function,where $v(\emptyset) = 0$.
    \end{block}
    \begin{itemize}
        \item The integer cooperative game is denoted by $G_I$ and set of all of them is denoted by $\mathcal{G}_I^N$
    \end{itemize}
    %\pause
    \begin{block}{Discrete cooperative game}
        A \textbf{discrete cooperative game} is an integer cooperative game $(N,v)$, where additionally we want the outcomes, i.e. payoff vectors, to be integers.
    \end{block}
    \begin{itemize}
        \item The integer cooperative game is denoted by $G_D$ and set of all of them is denoted by $\mathcal{G}_D^N$
    \end{itemize}
\end{frame}

%------------------------------------------------

%----------------------------------------------------------------------------------------
%	 SECTION 3
%----------------------------------------------------------------------------------------

\section{Our results}

%------------------------------------------------

\begin{frame}{Integer classes}
    \begin{itemize}
        \item Let $c \in \mathbb{N}_0$.
    \end{itemize}
    
    \begin{block}{$C$-tight games}
        A cooperative game $(N,v)$ is \textbf{c-tight} if it holds that
        \begin{displaymath}
            \forall S \subseteq N: 0 \leq v(S) \leq c \land v(N) = c.
        \end{displaymath}
    \end{block}

    \begin{itemize}
        \item For given $c$ there is $(c+1)^{2^n - 2}$ $c$-tight games.
    \end{itemize}
   
    \begin{block}{$C$-bounded games}
        A cooperative game $(N,v)$ is \textbf{c-bounded} if it holds that
        \begin{displaymath}
            \forall S \subseteq N: 0 \leq v(S) \leq c.
        \end{displaymath}
    \end{block}

    \begin{itemize}
        \item For given $c$ there is $(c+1)^{2^n - 1}$ $c$-bounded games.
    \end{itemize}

\end{frame}

%------------------------------------------------

%------------------------------------------------

\begin{frame}{Counting games}
    \begin{block}{Theorems $12$ and $14$}
        Let $c, n, k \in \mathbb{N}$ and $k \leq n$.
        The number of games with the following properties is given by

        \begin{itemize}
            \item $c$-tight integer $k$-games $\rightarrow$ $\binom{\binom{n}{k}-1+c}{c},$
            \item $c$-bounded integer $k$-games $\rightarrow$ $\sum_{i=0}^{c}\binom{\binom{n}{k}-1+i}{i}.$
        \end{itemize}
    \end{block}

    \begin{block}{Theorems $13$ and $15$}
        Let $c, n \in \mathbb{N}$.
        The number of games with the following properties is given by

        \begin{itemize}
            \item $c$-tight integer positive games $\rightarrow$ $\binom{2^{n}-2+c}{c},$
            \item $c$-bounded integer $k$-games $\rightarrow$ $\sum_{i=0}^{c}\binom{2^{n}-2+i}{i}.$
        \end{itemize}
    \end{block}
\end{frame}

%------------------------------------------------

%------------------------------------------------

\begin{frame}{Experiments}
    GRAPH
\end{frame}

%------------------------------------------------

%------------------------------------------------

\begin{frame}{Floor Shapley value}
    \begin{block}{Floor Shapley value}
        For an integer cooperative game $G_I \in \mathcal{G}_I^n$, the \textbf{Floor Shapley value} $\floorphi(G_I)$ is given by $\floorphi(G_I) = \floor{\phi(G_I)}$.
    \end{block}

    \begin{block}{Theorem $16$}
        The Floor Shapley value $\floorphi(v_I)$ satisfies the following properties for all integer games $(N,v_I),(N,w_I) \in \mathcal{G}_I^n$:
        \begin{enumerate}
            \item Axiom of near efficiency: $v_I(N) - n \leq \sum_{i \in N} \floorphi_i(v_I) \leq v_I(N)$,
            \item Axiom of symmetry: $\forall i,j \in N (\forall S \subseteq N \setminus \{i,j\}: v_I(S \cup i) = v_I(S \cup j)) \Rightarrow \floorphi_{i}(v_I) = \floorphi_{j}(v_I)$,
            \item Axiom of null player: $\forall i \in N(\forall S \subseteq N: v_I(S)=v_I(S \cup i)) \implies \floorphi_{i}(v_I) = 0$,
            \item Axiom of near additivity: $\floorphi(v_I+w_I) = \floor{\phi(v_I) + \phi(w_I)}$.
        \end{enumerate}
    \end{block}

\end{frame}

%------------------------------------------------

%------------------------------------------------

\begin{frame}{Efficient Floor Shapley value}
    \begin{itemize}
        \item The thought of preserving the efficiency...
        \item Redistribution of the remaining value...
    \end{itemize}

    \begin{block}{Efficient Floor Shapley value}
        For an integer cooperative game $G_I=(N,v_I) \in \mathcal{G}_I^n$, the \textbf{Efficient Floor Shapley value} $\phi^E(G_I)$ is defined as follows:
        \begin{enumerate}
            \item Compute the Floor Shapley value $\floorphi(G_I)$ and the Shapley value $\phi(G_I)$.
            \item Compute the weights $w_i = \phi_i(G_I) - \floorphi_i(G_I)$ for all $i \in N$.
            \item Sort the weights in descending order such that if multiple players have the same weight, then their ordering is uniformly random.
            \item Each player receives his Floor Shapley value. Additionally, the top $k$ players, where $k = v_I(N) - \sum_{i \in N} \floorphi_i(v_I) = w(N)$, receive one extra unit.
        \end{enumerate}
    \end{block}
\end{frame}

%------------------------------------------------

%------------------------------------------------

\begin{frame}{Efficient Floor Shapley value - properties}

    \begin{block}{Theorem $17$}
        The Efficient Floor Shapley value $\phi^E$ satisfies the following properties for all integer games $(N,v_I) \in \mathcal{G}_I^n$:
        \begin{enumerate}
            \item Axiom of efficiency: $\sum_{i \in N}\phi_{i}^E(v_I) = v_I(N)$,
            \item Axiom of expected symmetry: $\forall i,j \in N (\forall S \subseteq N \setminus \{i,j\}: v_I(S \cup i) = v_I(S \cup j)) \Rightarrow \E[\phi^E_{i}(v_I)] = \E[\phi^E_{j}(v_I)]$,
            \item Axiom of null player: $\forall i \in N(\forall S \subseteq N: v_I(S)=v_I(S \cup i)) \implies \phi_{i}^E(v_I) = 0$.
        \end{enumerate}
    \end{block}

    \begin{itemize}
        \item No axiom of additivity % TODO: v nekterych det a v sub random - example 
    \end{itemize}
\end{frame}

%------------------------------------------------

%------------------------------------------------

\begin{frame}{Probabilistic Efficient Floor Shapley value}
    \begin{itemize}
        \item Same thought, different approach...
    \end{itemize}

    \begin{block}{Probabilistic Efficient Floor Shapley value}
        For an integer cooperative game $G_I \in \mathcal{G}_I^n$, the \textbf{Probabilistic Efficient Floor Shapley value} $\phi^{\E}(v)$ is defined as follows:
        \begin{enumerate}
            \item Compute the Floor Shapley value $\floorphi(G_I)$ and the Shapley value $\phi(G_I)$.
            \item Compute the remainders $\tilde{p}_i = \phi_i(G_I) - \floorphi_i(G_I)$ for all $i \in N$.
            \item Compute the probabilities $p_i = \frac{\tilde{p}_i}{\sum_{j \in N} \tilde{p}_j}$ for all $i \in N$.
            \item Each player receives his Floor Shapley value and additionally, each unit of the remainder with probability $p_i$, i.e., each unit of $\tilde{p}(N)=\sum_{j \in N} \tilde{p}_j$ is given to player $i$ with probability $p_i$.
        \end{enumerate}
    \end{block}
\end{frame}

%------------------------------------------------

%------------------------------------------------

\begin{frame}{Probabilistic Efficient Floor Shapley value - properties}

    \begin{block}{Theorem $18$}
        The Probabilistic Efficient Floor Shapley value $\phi^{\E}$  satisfies the following properties for all integer games $(N,v_I),(N,w_I) \in \mathcal{G}_I^n$:
        \begin{enumerate}
            \item The expected value is the same as the Shapley value: $\E[\phi^{\E}(v_I)] = \phi(v_I)$,
            \item Axiom of efficiency: $\sum_{i \in N}\phi_{i}^{\E}(v_I) = v_I(N)$,
            \item Axiom of expected symmetry: $\forall i,j \in N (\forall S \subseteq N \setminus \{i,j\}: v_I(S \cup i) = v_I(S \cup j)) \Rightarrow \E[\phi_{i}^{\E}(v_I)] = \E[\phi_{j}^{\E}(v_I)]$,
            \item Axiom of null player: $\forall i \in N(\forall S \subseteq N: v_I(S)=v_I(S \cup i)) \implies \phi_{i}^{\E}(v_I) = 0$,
            \item Axiom of expected additivity: $\E[\phi^{\E}(v_I+w_I)] = \E[\phi^{\E}(v_I)] + \E[\phi^{\E}(w_I)]$.
        \end{enumerate}
    \end{block}

\end{frame}

%------------------------------------------------

\section{Thank you for your attention}

\end{document}
