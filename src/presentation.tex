%%%%%%%%%%%%%%%%%%%%%%%%%%%%%%%%%%%%%%%%%
% Focus Beamer Presentation
% LaTeX Template
% Version 1.0 (8/8/18)
%
% This template has been downloaded from:
% http://www.LaTeXTemplates.com
%
% Original author:
% Pasquale Africa (https://github.com/elauksap/focus-beamertheme) with modifications by
% Vel (vel@LaTeXTemplates.com)
%
% Template license:
% GNU GPL v3.0 License
%
% Important note:
% The bibliography/references need to be compiled with bibtex.
%
%%%%%%%%%%%%%%%%%%%%%%%%%%%%%%%%%%%%%%%%%

%----------------------------------------------------------------------------------------
%	PACKAGES AND OTHER DOCUMENT CONFIGURATIONS
%----------------------------------------------------------------------------------------
\documentclass{beamer}
%\documentclass[handout]{beamer}

\usepackage{mathtools}
\usepackage{comment}

\usetheme{focus} % Use the Focus theme supplied with the template
% Add option [numbering=none] to disable the footer progress bar
% Add option [numbering=fullbar] to show the footer progress bar as always full with a slide count

% Uncomment to enable the ice-blue theme
%\definecolor{main}{RGB}{92, 138, 168}
%\definecolor{background}{RGB}{240, 247, 255}

\definecolor{mygreen}{RGB}{0, 128, 0}
\definecolor{myblue}{RGB}{51, 51, 204}
\definecolor{myviolet}{RGB}{255, 0, 255}

\newcommand{\red}[1]{\textcolor{red}{#1}}
\newcommand{\green}[1]{\textcolor{mygreen}{#1}}
\newcommand{\blue}[1]{\textcolor{myblue}{#1}}
\newcommand{\violet}[1]{\textcolor{myviolet}{#1}}

\newcommand{\R}{\mathbb{R}}
\newcommand{\Z}{\mathbb{Z}}
\newcommand{\Rn}[1][n]{\mathbb{R}^{#1}}
\newcommand{\bx}{\textbf{x}}
\newcommand{\by}{\textbf{y}}
\newcommand{\bz}{\textbf{z}}
\newcommand{\bu}{\textbf{u}}
\newcommand{\bv}{\textbf{v}}
\newcommand{\I}{\mathcal{I}}


%------------------------------------------------

\usepackage{booktabs} % Required for better table rules
%\usepackage{slashbox}
%\usepackage{ulem,centernot}


%----------------------------------------------------------------------------------------
%	 TITLE SLIDE
%----------------------------------------------------------------------------------------

\title{Integer cooperative game theory}

\author{Richard Mužík}

\titlegraphic{\hspace*{5cm}\includegraphics[scale=0.5]{../img/logo-en.pdf}} % Optional title page image, comment this line to remove it

\institute{richard@imuzik.cz}


%------------------------------------------------

\begin{document}

%------------------------------------------------

\begin{frame}
	\maketitle % Automatically created using the information in the commands above
\end{frame}

%----------------------------------------------------------------------------------------
%	 SECTION 1
%----------------------------------------------------------------------------------------

\section{Motivations and introduction} % Section title slide, unnumbered

%------------------------------------------------

\begin{frame}{Game theory - quick reminder}
    \begin{block}{Cooperative game}
        A \textbf{cooperative game} is an ordered pair $(N,v)$, where $N=\{1,\dots,n\}$ is a finite set of players and $v\colon 2^N \to \mathbb{R}$ is the characteristic function. Further, $v(\emptyset) = 0$.
    \end{block}
    \begin{itemize}
        \item $S \subseteq N$ ... coalition
        \item $v(S)$ ... values of coalition
    \end{itemize}
    Goal: \textit{To distribute the value of the game among players}
    \begin{itemize}
	    \item \textbf{Payoff vector} $\bx \in \Rn$
	    \begin{itemize}
	        \item $x_i$ represents payoff of player $i$
	    \end{itemize}
	    \item Vector $\bx \in \Rn$ is \textbf{efficient}, if $\sum_{i \in N}x_i = v(N)$
	    \begin{itemize}
	        \item Usually, we distribute $v(N)$
	    \end{itemize}
	    \item Vector $\bx \in \Rn$ is \textbf{individually rational}, if $x_i \geq v(i)$
	    \begin{itemize}
	        \item players prefer $x_i$ over $v(i)$
	    \end{itemize}
	\end{itemize}
\end{frame}

%------------------------------------------------


%----------------------------------------------------------------------------------------
%	 SECTION 2
%----------------------------------------------------------------------------------------

\section{Definitions} % Section title slide, unnumbered

%------------------------------------------------

\begin{frame}{Discrete cooperative game theory}
    \begin{block}{Integer cooperative game}
        An \textbf{integer cooperative game} is an ordered pair $(N,v)$, where $N=\{1,\dots,n\}$ is a finite set of players and $v\colon 2^N \to \mathbb{Z}$ is the characteristic function,where $v(\emptyset) = 0$.
    \end{block}
    \begin{itemize}
        \item The integer cooperative game is denoted by $G_I$ and set of all of them is denoted by $\mathcal{G}_I^N$
    \end{itemize}
    %\pause
    \begin{block}{Discrete cooperative game}
        A \textbf{discrete cooperative game} is an integer cooperative game $(N,v)$, where additionally we want the outcomes, i.e. payoff vectors, to be integers.
    \end{block}
    \begin{itemize}
        \item The integer cooperative game is denoted by $G_D$ and set of all of them is denoted by $\mathcal{G}_D^N$
    \end{itemize}
\end{frame}

%------------------------------------------------

\section{Thank you for your attention}

\end{document}
